\documentclass[compress,handout,10pt]{beamer}

\newlength{\wideitemsep}
\setlength{\wideitemsep}{\itemsep}
\addtolength{\wideitemsep}{100pt}
\let\olditem\item
\renewcommand{\item}{\setlength{\itemsep}{0.5\baselineskip}\olditem}

\usetheme{Warsaw}
\usecolortheme{default}
\usefonttheme[onlymath]{serif}

\usepackage{float}
\floatstyle{boxed}
\usepackage{colortbl}
\usepackage{mathpazo}
\usepackage{graphicx}
\usepackage{movie15}
\usepackage{bm}
\usepackage{verbatim}
\usepackage{comment}
\usepackage{caption}
\usepackage{subcaption}
\captionsetup[subfigure]{labelformat=empty}
\captionsetup[figure]{labelformat=empty}

\newcommand{\mygreen}{\color{green!50!black}}
\newcommand{\myblue}{\color{blue}}
\newcommand{\myred}{\color{red}}
\newcommand{\mycolor}{\color{red}{c}\color{blue}{o}\color{green}{l}\color{orange}{o}\color{cyan}{r}}
\newcommand{\mysize}{\scriptsize{s}\small{i}\normalsize{z}\Large{e}}
\newcommand{\myshape}{\textcircled{s}\textit{h}\texttt{a}\textsf{p}\textsc{e}}

\xdefinecolor{titlecolor}{rgb}{.855,.647,.125}
\setbeamercolor{frametitle}{fg=titlecolor}
\setbeamerfont{frametitle}{series=\bfseries}
\setbeamercolor{normal text in math text}{parent=math text}

\setbeamertemplate{navigation symbols}{} %gets rid of navigation symbols
\setbeamertemplate{footline}[frame number]
\beamertemplateshadingbackground{yellow!10}{green!5}

\title{{\color{yellow} \LARGE Modeling the Sociodynamics of Applause\newline} }

\subtitle{{\color{white} \large Midterm Presentation} }

\author{ 
%    \vspace{5pt}
    {\bf{Participants:}} \\ 
Ahmed~Aly \\ 
    \vspace{5pt}
} 
\institute{{Department of Applied Mathematics and Statistics, Johns Hopkins University}\\
{Department of Sociology, Johns Hopkins University}}

\date{\mygreen  \today} 

\begin{document}

\begin{frame}[plain]
    \titlepage
\end{frame}

\begin{frame}
    \frametitle{Outline}
    \tableofcontents
\end{frame}

\section{Problem Description}

\begin{frame}


\begin{figure}[h]
    \begin{center}
        \includegraphics[width=\textwidth]{../images/congress.jpg}
    \end{center}
    \caption{Congress}
    \label{fig:congress}
\end{figure}

\end{frame}


\begin{frame}
\begin{figure}[h]
    \begin{center}
        \includegraphics[width=\textwidth]{../images/state_union.jpg}
    \end{center}
    \caption{State of the Union}
    \label{fig:stateunion}
\end{figure}

\end{frame}



\begin{frame}
   \frametitle{Value and Application}
    Ideally, the Model:
    \vspace{7pt}
	\begin{enumerate}
		\item Measures approval/acceptance of subject,\newline
		\item Can be applied to get a create a full blown applause,\newline
		\item Describes the transfer of ideas and the rate of approval,\newline
	\end{enumerate}	
\end{frame}

\begin{frame}
  \frametitle{Meet the Sponsors}
   Because the project is in research phase the sponsors have been chosen to be in an academic setting
	\begin{enumerate}
		\item Department of Applied and Mathematics and Statistics at JHU\newline \newline
	is well known for its multi-faceted and verastile research as well as its industrial connections		
		\item Department of Sociology at JHU\newline \newline
	is well known for its research in group psychology, social interactions, and group dynamics
	\end{enumerate} 
\vspace{7pt}

Once a model is produced more industrial sponsors such as google, facebook, HBO, etc. can be added.

\end{frame}

\section{Deliverables}

\begin{frame}
   \frametitle{Goals}
   The main goal is to model the dynamics applause in an audience and to establish the critical mass needed to start a full blown applause.\newline\newline
 Deliverables:
 \vspace{7pt}
	\begin{enumerate}
		\item A simple model of the individual,
		\item An integrated model of the crowd,
		\item A simulation to demostrate behavior given parameter changes,
		\item Technical reports and presentations summarizing the work.
	\end{enumerate}



\end{frame}


\section {Mathematical Background and Related Work}

\begin{frame}
  \frametitle{Related Work}
	\begin{enumerate}
		\item Diffusion of Innovations by Everett Rogers \newline\newline
		Details the behavior and adoption of innovations and categorizes adopters.
\begin{figure}[h]
    \begin{center}
        \includegraphics[width=\textwidth]{../images/DiffusionOfInnovation.png}
    \end{center}
    \caption{Categories of Innovativeness}
    \label{diffusion}
\end{figure}
	\end{enumerate}
\end{frame} 

\begin{frame}
  \frametitle{Key Observations}
	\begin{enumerate}
		\item Members in the crowd are compelled to clap if crowd is clapping,\newline
		\item The greater the intensity and duration of applause the greater the approval,\newline
		\item After a full blown applause, there is a wait period in which clapping would be too late and full applause can not be generated,\newline
		\item Willingness of individual members to clap depends on percieved intensity, stimulus, emotional state, and resistance to the crowd.
	\end{enumerate}

\end{frame}

\begin{frame}
   \frametitle{Key Assumptions}
	\begin {enumerate}
		\item Stimulus (speech, opinion, performance, etc.)  is average and constant,\newline
		\item Clapping is only a result of a positive response other reasons are disregarded,\newline
		\item Psychological state of individual is stochastic
	\end{enumerate}	
\end{frame}

\section{Approach}
\begin{frame}
\frametitle{First Objective}
We will model individual as a open loop system.\newline\newline
3 factors to consider: \newline
	\begin{enumerate}
		\item mood (happy, sad, angry, etc.),\newline \newline
		\item resistance and connectivity with the crowd,\newline\newline
		\item social inhibition.		
	\end{enumerate}
\end{frame}

\begin{frame}
\frametitle{First Objective}
Let $N$ be the population that claps, $I$ be intensity and $T$ be duration then we can think of the individual as:\newline\newline\newline

\begin{figure}[h]
    \begin{center}
        \includegraphics[width=\textwidth]{../images/Indivi.jpg}
    \end{center}
    \caption{Scheme of the Individual}
    \label{fig:individual}
\end{figure}

\end{frame}

\begin{frame}
\frametitle{Second Objective}
The second objective is to integrate the indivual models to examine the behavior of the population.\newline \newline
The critical mass/threshold needed to start a full-blown applause will be determined.
\end{frame}



\begin{frame}
    \frametitle{Getting People on the Bandwagon!}
    Measuring Approval and Acceptance:
    \vspace{7pt}
             \begin{enumerate}
                 \item Work Statement
                 \item Midterm Presentation
                 \item Progress Report
                 \item Final Presentation
                 \item Final Report
             \end{enumerate}
\end{frame}

\begin{frame}
    \frametitle{Programmings in this class}
    \begin{enumerate}
        \item \LaTeX: 
            \begin{enumerate}
                \item \texttt{moderncv}
                \item \texttt{beamer}
                \item \texttt{article}
                \item \texttt{tikz}
            \end{enumerate}
        \item R:
            \begin{enumerate}
                \item \texttt{tikzDevice}
                \item \texttt{lm}
            \end{enumerate}
        \item Git
            \begin{enumerate}
                \item \texttt{git init .}
            \end{enumerate}
    \end{enumerate}
\end{frame}

\section{Principles}
\begin{frame}
    \frametitle{Seven Basic Principles}
     \begin{enumerate}
         \item Set the context 
         \item Choose effective examples and analogies
         \item Choose vocabulary to suit your readers
         \item Decide whether to present \#s in text, tables, or figures
         \item Report and interpret \#s in the text
         \item Specify the direction \emph{and} size of an association between variables
         \item For many \#s, summarize overall pattern 
     \end{enumerate}
\end{frame}

\section{Tools}
\begin{frame}
    \frametitle{Creating Effective Tables}
\end{frame}

\section{Arguments from Scale}

\begin{frame}
    \frametitle{Example: Cost of Packaging}
\end{frame}

\section{Graphical Methods}
\begin{frame}
    \frametitle{Example: The Nuclear Mission Arms Race}
\end{frame}

\section{Basic Optimization}
\begin{frame}
    \frametitle{Example: Maintaining Inventory}
\end{frame}

\end{document}
